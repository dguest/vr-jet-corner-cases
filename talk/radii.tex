\documentclass[xcolor={table}]{beamer}
\usepackage[utf8]{inputenc}
\useoutertheme{infolines}
\usepackage{transparent}

%%%%% Title and Author %%%%%
\title[VR-Tagging]{Flavor Tagging VR Jets}
\subtitle{What could possibly go wrong?}
\author[dguest@cern.ch]{Dan~Guest}
\institute[UCI]{UC~Irvine}

\definecolor{UCIBlue}{RGB}{0,100,164}
\definecolor{UCIYellow}{RGB}{255,210,0}
\definecolor{UCIYDull}{RGB}{247,235,95}
\definecolor{UCIBDull}{RGB}{106,162,184}
\definecolor{UCIBDark}{RGB}{27,61,109}
\usecolortheme{crane}
\setbeamercolor*{palette primary}{use=structure,fg=white,bg=UCIBlue}
\setbeamercolor*{palette secondary}{use=structure,fg=white,bg=UCIYellow}
\setbeamercolor*{palette tertiary}{use=structure,fg=white,bg=UCIBDark}

\definecolor{dg}{rgb}{0.0,0.5,0.0}

\setbeamertemplate{enumerate items}[default]
\setbeamertemplate{navigation symbols}{}
\setbeamercovered{transparent}
\usefonttheme{serif} % default family is serif
\newcommand{\um}{\mu \mathrm{m}}
\newcommand{\mm}{\mathrm{mm}}
\newcommand{\cm}{\mathrm{cm}}
\newcommand{\gev}{\mathrm{GeV}}
\newcommand{\mev}{\mathrm{MeV}}
\newcommand{\tev}{\mathrm{TeV}}
\newcommand{\T}{\mathrm{T}}
\newcommand{\pt}{p_{\mathrm{T}}}
\newcommand{\met}{E_{\mathrm{T}}^{\mathrm{miss}}}
\newcommand{\supp}[1]{\tilde{#1}}
\newcommand{\neut}{\supp{\chi}_1^0}
\newcommand{\cha}{\supp{\chi}_{1}^{\pm}}

\newcommand{\graphic}[2][0.99]{\includegraphics[width=#1\textwidth]{#2}}
\newcommand{\backupbegin}{
   \newcounter{framenumberappendix}
   \setcounter{framenumberappendix}{\value{framenumber}}
}
\newcommand{\backupend}{
   \addtocounter{framenumberappendix}{-\value{framenumber}}
   \addtocounter{framenumber}{\value{framenumberappendix}}
}
\newcommand{\link}[2]{\underline{\href{#2}{#1}}}
\newcommand{\arxiv}[1]{\link{arXiv:#1}{http://arxiv.org/abs/#1}}
\newcommand{\twocol}[3][0.5]{
  \newdimen\scwid
  \scwid=\dimexpr\textwidth-#1\textwidth\relax
  \begin{columns}
    \begin{column}{#1\textwidth}#2\end{column}
      \begin{column}{\scwid}#3\end{column}
  \end{columns}
}
\newcommand{\cent}[1]{\begin{center}#1\end{center}}

% -- set graphics path --
\graphicspath{{inkfigs/}{rafaelfigs/}{genfigs/}}

% _________________________________________________________________________
% main document starts here

\begin{document}

\begin{frame}
\maketitle
\end{frame}

\begin{frame}
  \frametitle{Introduction: Many ``R''s}
  \begin{center}
  \graphic{radii-vs-pt.pdf}
  \end{center}
\end{frame}

\begin{frame}
  \frametitle{Let's Take Two Jets}
  \begin{center}
  \graphic{radii-vs-pt-jets.pdf}
  \end{center}
\end{frame}

\begin{frame}
  \frametitle{Now make them concentric}
  \twocol{
    \graphic{arrows.pdf}
  }{
    \graphic{radii-vs-pt-jets.pdf}
    \begin{itemize}
    \item If $R_{2} > \Delta R (1,2) +  R_{1}$, \colorbox{red}{Jet 1} is enclosed by \colorbox{red!50!}{Jet 2}
    \item If $R_{1} > \Delta R (1,2)$, the center of \colorbox{red!50!}{Jet 2} is ``in'' \colorbox{red}{Jet 1}
    \end{itemize}
  }
\end{frame}

\begin{frame}
  \frametitle{How often does this happen?}
  \framesubtitle{Thanks To Rafael Teixeira De Lima for plots}
  \twocol{
    \begin{center}
      Jet 1 in jet 2 (ok)
      \graphic{logvr1mvr2_m.pdf}
    \end{center}
  }{
    \begin{center}
      Jet 2 in jet 1 (bad)
      \graphic{logvr1_m.pdf}
    \end{center}
  }
  \begin{itemize}
  \item Enclosed high $\pt$ jet is quite common
    \begin{itemize}
    \item This might be ok
    \end{itemize}
  \item Enclosed low $\pt$ jet is less so
    \begin{itemize}
    \item Good news, because \textbf{this is the case we worry about}
    \end{itemize}
  \end{itemize}
\end{frame}


\begin{frame}
  \frametitle{What could go wrong?}
  \twocol{
    \graphic{vr.pdf}
  }{
    \begin{itemize}
    \item \colorbox{red}{Jet 1 forms high $\pt$ core}
    \item 10\% of the radiation isn't contained
    \item \colorbox{red!50!}{Jet 2 is forms}
    \end{itemize}
  }
  \begin{itemize}
  \item From a ``jet'' perspective this is actually quite nice
  \item Let's try flavor tagging!
  \end{itemize}
\end{frame}

\begin{frame}
  \frametitle{Things Get Weird (part 1): Labeling}
  \twocol{
    \graphic{labeling.pdf}
  }{
    \begin{itemize}
    \item If the $b$-hadron points right, label
      \colorbox{blue!25!}{high $\pt$} jet
    \item If it points left, label \colorbox{blue!10!}{softer one}
    \item Very ``IRC unsafe''!
    \item Label is \emph{undefined} if $\Delta R = 0$
    \end{itemize}
  }
\end{frame}

\begin{frame}
  \frametitle{Things Get Weird (part 2): Track Association}
  \twocol{
    \graphic{tracking.pdf}
  }{
    \begin{itemize}
    \item Same drill as labeling
    \item If the track points right, assign to
      \colorbox{green!25!}{high $\pt$} jet
    \item If it points left, assign to \colorbox{green!10!}{softer one}
    \item The upshot: It's consistent with labeling
    \end{itemize}
  }
\end{frame}

\begin{frame}
  \frametitle{One solution: Ghost Everything}
  \framesubtitle{Ghost associate $b$-hadrons and tracks}
  \twocol{
    \graphic{vr.pdf}
  }{
    \begin{itemize}
    \item No more confusing edge-cases
    \item Simpler optimization (one parameter)
    \item It \emph{should} work
    \item Current association is pre--Run-1
    \end{itemize}
  }
  \begin{itemize}
  \item The code is (mostly) in place (thanks to Jie Yu, Chris Pollard)
  \end{itemize}
\end{frame}

\begin{frame}
  \frametitle{Conclusions and Open Questions}
  \begin{itemize}
  \item VR jets have some weird edge cases
    \begin{itemize}
    \item Especially in boosted flavor tagging
    \end{itemize}
  \item We could ``solve'' them by moving away from cone-based labels and association
  \end{itemize}
  \begin{block}{Is it worth it?}
    \begin{itemize}
    \item \textcolor{dg}{Might prevent surprises later on}
    \item There were implementation issues before (solved now)
    \item \textcolor{red}{Seems to be limited to only a few cases (so far)}
    \end{itemize}
  \end{block}
\end{frame}

% put slides

\end{document}
